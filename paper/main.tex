%%%%%%%%%%%%%%%%%%%%%%%%%%%%%%%%%%%%%%%%%%%%%%%%%%%%%%%%%%%%%%%%%%%%%%%%%%%%%%%%%%%%%%%%%%%%%%%%%%%%%%%%%%%%%%%%%%%%%%%%%%%%%%%%%%%%%%%%%%%%%%%%%%%%%%%%%%%
% This is just an example/guide for you to refer to when submitting manuscripts to Frontiers, it is not mandatory to use Frontiers .cls files nor frontiers.tex  %
% This will only generate the Manuscript, the final article will be typeset by Frontiers after acceptance.                                                 %
%                                                                                                                                                         %
% When submitting your files, remember to upload this *tex file, the pdf generated with it, the *bib file (if bibliography is not within the *tex) and all the figures.
%%%%%%%%%%%%%%%%%%%%%%%%%%%%%%%%%%%%%%%%%%%%%%%%%%%%%%%%%%%%%%%%%%%%%%%%%%%%%%%%%%%%%%%%%%%%%%%%%%%%%%%%%%%%%%%%%%%%%%%%%%%%%%%%%%%%%%%%%%%%%%%%%%%%%%%%%%%

%%% Version 3.1 Generated 2015/22/05 %%%
%%% You will need to have the following packages installed: datetime, fmtcount, etoolbox, fcprefix, which are normally inlcuded in WinEdt. %%%
%%% In http://www.ctan.org/ you can find the packages and how to install them, if necessary. %%%

\documentclass[english]{frontiers/frontiersSCNS} % for Science, Engineering and Humanities and Social Sciences articles
%\documentclass{frontiersHLTH} % for Health articles
%\documentclass{frontiersFPHY} % for Physics and Applied Mathematics and Statistics articles

\usepackage[mode=buildnew]{standalone}
\usepackage{tikz}
%\setcitestyle{square}
\usepackage{url,lineno,microtype}
\usepackage[toc,nomain,acronym,shortcuts,translate=false]{glossaries}
\usepackage[colorlinks=true,linkcolor=black, citecolor=black!80, urlcolor=black!80]{hyperref}
\usepackage{doi}
\usepackage[onehalfspacing]{setspace}

\def\keyFont{\fontsize{7}{9}\helveticabold }
\def\firstAuthorLast{Esteban {et~al.}} %use et al only if is more than 1 author
\def\Authors{Oscar Esteban\,$^{1,2,*}$, Emmanuel Caruyer\,$^{3}$, Alessandro Daducci\,$^{4}$, Meritxell Bach-Cuadra\,$^{4,5}$,%
Mar\'ia-J. Ledesma-Carbayo\,$^{1,2}$ and Andres Santos\,$^{1,2}$}

\def\Address{%
$^{1}$Biomedical Image Technologies (BIT), ETSI Telecom., Universidad Polit\'ecnica de Madrid, Madrid, Spain \\
$^{2}$Centro de Investigaci\'on Biom\'edica en Red en Bioingenier\'ia, Biomateriales y Nanomedicina (CIBER-BBN), Spain \\
$^{3}$CNRS, IRISA (UMR 6074), VisAGeS research group, Rennes, France \\
$^{4}$Signal Processing Laboratory (LTS5), \'Ecole Polytechnique F\'ed\'erale de Lausanne (EPFL), Lausanne, Switzerland \\
${^5}$Dept. of Radiology, CIBM, University Hospital Center (CHUV) and University of Lausanne (UNIL), Lausanne, Switzerland %
}
% The Corresponding Author should be marked with an asterisk
% Provide the exact contact address (this time including street name and city zip code) and email of the corresponding author
\def\corrAuthor{Oscar Esteban}
\def\corrAddress{Biomedical Image Technologies (BIT), ETSI Telecomunicaci\'on, Av. Complutense 30, C203, E28040 Madrid, Spain}
\def\corrEmail{phd@oscaresteban.es}

% -*- root: main.tex -*-
% @Author: Oscar Esteban
% @Date:   2015-06-16 15:45:32
% @Last Modified by:   Oscar Esteban
% @Last Modified time: 2015-06-17 16:28:24

\newacronym{dmri}{dMRI}{diffusion MRI}
\newacronym{hcp}{HCP}{Human Connectome Project}

\makeglossaries


\newcommand{\e}[1]{\ensuremath{\;\cdot\,}10\ensuremath{^{#1}}}
\newcommand*\glslines[2][]{\glslink[#1]{#2}{\glsdisp{#2}{\acrlong{#2} --\glsentryshort{#2}--}}}

\begin{document}

\onecolumn
\firstpage{1}

\title[Diffantom]{Diffantom: a software layer to build digital phantoms of whole-brain diffusion MRI}

\author[\firstAuthorLast ]{\Authors} %This field will be automatically populated
\address{} %This field will be automatically populated
\correspondance{} %This field will be automatically populated

\extraAuth{}% If there are more than 1 corresponding author, comment this line and uncomment the next one.
%\extraAuth{corresponding Author2 \\ Laboratory X2, Institute X2, Department X2, Organization X2, Street X2, City X2 , State XX2 (only USA, Canada and Australia), Zip Code2, X2 Country X2, email2@uni2.edu}


\maketitle

\linenumbers

\section*{Introduction}
Fiber tracking on \gls*{dmri} data has become an important application in the in-vivo investigation
  of the structural configuration of fiber bundles at the macro-scale.
The final outcome of such applications, called tractography, is fundamental to gain information about
  \gls*{wm} morphology in neurosurgical planning \citep{golby_interactive_2011},
  post-surgery evaluations \citep{toda_utility_2014},
  and the study of neurological diseases such as multiple sclerosis and Alzheimer's disease
  \citep{chua_diffusion_2008}.
Tractography is also crucial in the analysis of structural brain networks using graph theory.
These analyses have been successfully applied in the definition of the unique subject-wise patterns of connectivity
  \citep{sporns_human_2005}, the assessment of neurological diseases \citep{griffa_structural_2013}, and in the
  study of the link between structural and functional connectivity \citep{messe_predicting_2015}.
However, the development of the field is limited by the lack of gold standards to test and compare the
  wide range of algorithms and processing methodologies available.
In this report we present \emph{diffantom}, an \emph{in-silico} dataset to assess tractography pipelines using
  real data as source of whole-brain microstructural information.
We also provide a full software layer to produce \emph{diffantoms} from real \gls*{dmri} datasets
  drawn from the \gls*{hcp}.
The \emph{diffantom} enables the following evaluations of standard connectivity pipelines:
\begin{itemize}
  \item Comparisons of the impact of different diffusion sampling schemes on the local micro-structure
  model of choice and the subsequent global tractography outcome.
  \item Investigations on the effects of the different artifacts (noise, \acrlong{pve} and \glslines{csf} contamination,
    susceptibility-derived warping, Eddy-currents-derived distortions, etc.) at both local and global levels.
  \item Characterizing the \acrfull*{roc} of connectivity pipelines.
  \item Generation of synthetic connectivity networks to test graph theory analyses.
  \item Simulation of pathological brains (e.g. tumors as in \cite{kaus_simulation_2000}).
\end{itemize}


\section*{Background}
The assessment of tractography from \gls*{dmri} data is challenging due to
  the lack of a gold standard.
Large efforts have been devoted to the development of physical phantoms
  \citep{lin_validation_2001,campbell_flowbased_2005,perrin_validation_2005,fieremans_simulation_2008,tournier_resolving_2008}.
\cite{cote_tractometer_2013} conducted a thorough review of tractography methodologies using the
  so-called \emph{FiberCup} phantom \citep{poupon_new_2008,fillard_quantitative_2011}.
These phantoms are appropriate to evaluate the angular resolution in fiber crossings and accuracy of
  direction-independent scalar parameters in very simplistic geometries.
Since the complexity of whole-brain tractography is unaccountable with current materials and proposed
  methodologies to build physical phantoms, digital simulations are increasingly popular.
Early digital phantoms started with simulation of simple geometries as well
  \citep{basser_in_2000,goessl_fiber_2002,tournier_limitations_2002,leemans_mathematical_2005}
  to evaluate resolution of fiber crossings.
These tools generally implemented the multi-tensor model \citep{alexander_analysis_2001,tuch_high_2002}
  to simulate fiber crossing, fanning, kissing, etc.
\cite{close_software_2009} presented the \emph{Numerical Fibre Generator}, a software to simulate
  spherical shapes filled with digital fiber-tracts.
\cite{caruyer_Phantomas_2014} proposed \emph{Phantomas} to simulate any kind of analytic geometry
  inside a sphere.
\emph{Phantomas} models diffusion by a restricted and a hindered compartment, similar to
  \citep{assaf_composite_2005}.
\cite{wilkins_fiber_2014} proposed a whole-brain simulated phantom derived from voxel-wise orientation
  of fibers averaged from real \gls*{dmri} scans and the multi-tensor model with a compartment of
  isotropic diffusion.
Recently, \cite{neher_fiberfox_2014} proposed \emph{FiberFox}, a visualization software to develop
  complex geometries and their analytical description.
Once the geometries are obtained, the software generates the corresponding \gls*{dmri} signal with a
  methodology very close to that implemented in \emph{Phantomas}.


\section*{Data description}

\noindent\textbf{\textit{Data Generation\textcolon}\label{sec:data_generation}}
The methods to generate \emph{diffantoms} are rooted in the work of \cite{wilkins_fiber_2014}, and
  described in \autoref{fig:figure1}.
\emph{Diffantom} integrates two major improvements with respect to \citep{wilkins_fiber_2014}.
First, since we use a dataset from the \gls*{hcp} as input, data are corrected for the most relevant distortions.
\cite{wilkins_fiber_2014} explicitly state that their original data were not subjected to certain distortion
  corrections, and thus, generated data are affected correspondingly.
Second improvement is a more advanced signal model to generate the phantom using
  \emph{Phantomas} \citep{caruyer_Phantomas_2014}.

The simulation process begins with generating a microstructural model from real data (\autoref{fig:figure1}).
On one hand, the software requires up to five volume fractions $\{T_{1\,\cdots\,5}\}$ corresponding to the free-diffusion
  compartments that will be simulated.
These isotropic fraction maps use the \emph{5TT format} established with the latest version 3.0 of
  \emph{MRTrix} \citep{tournier_mrtrix_2012}, that describes the structure of the brain in the following
  order: \gls*{cgm}, \gls*{dgm}, \gls*{wm}, \gls*{csf}, and abnormal tissue.
Since we simulated a healthy subject, the last fraction map is empty and can be omitted.
On the other hand, the hindered-diffusion compartments are specified by up to three volume fractions $\{F_{1\,\cdots\,3}\}$
  of three single fiber populations per voxel along with their corresponding direction maps $\{V_{1\,\cdots\,3}\}$.

The process to obtain the microstructural model is described as follows:
1) The fiber orientation maps $\{V_{1\,\cdots\,3}\}$ and their corresponding fiber-fraction maps are
  obtained using the ball-and-stick model for multi-shell data of BEDPOSTX \citep{jbabdi_modelbased_2012}
  on the \gls*{dmri} data of a certain \gls*{hcp} subject.
2) We obtain a \gls*{fa} map after fitting a tensor model with \emph{MRTrix}.
3) The original fiber-fractions and the \gls*{fa} map are smoothed with a non-local means filter included
  in \emph{dipy} \citep{garyfallis_dipy_2011}.
4) By combining with \emph{MRTrix} the volume fraction maps computed with FSL FAST \citep{zhang_segmentation_2001} and
  FSL FIRST \citep{patenaude_bayesian_2011}, we obtain the macro-structural 5TT map of the subject.
5) The objects obtained in the previous steps are combined as described in the \nameref{sec:appendix} to generate the
  microstructural model, presented in \autoref{fig:figure1} box A.

Once a micro-structural model of the volume has been synthesized, a map of \glspl{fod} is computed using spherical
  convolution of the single fiber response and the fiber orientation maps $\{V_{1\,\cdots\,3}\}$, weighted by the
  corresponding fiber-fractions $\{F_{1\,\cdots\,3}\}$.
A close-up showing how the \glspl{fod} map looks is presented in \autoref{fig:figure1}.
The single fiber response is a Gaussian diffusion tensor with axial symmetry and eigenvalues $\lambda_1=$ 2.2\e{-3}
  and $\lambda_{2,3}=$ 0.2\e{-3} [mm$^2$s$^{-1}$].
The resulting \glspl{fod} map is then combined with the isotropic compartments corresponding to $\{T_{1\,\cdots\,5}\}$
  before generating the signal using \emph{Phantomas} \citep{caruyer_Phantomas_2014}.
By default, diffusion data are generated using a scheme of 100 directions distributed in one shell with uniform
  coverage \citep{caruyer_design_2013}.
Custom one- or multi-shell schemes can be generated supplying the tables of corresponding $b$-vectors and $b$-values.
Rician noise is also included in \emph{Phantomas}, and can be set by the user.
The default value for \gls*{snr} is preset to 90.0.
\Gls*{csf} was simulated with a unique compartment with isotropic diffusivity $D_{CSF}$ of 3.0\e{-3} [mm$^2$s$^{-1}$].
Conversely, \Gls*{wm} and \gls*{gm} included one isotropic compartment with $D_{WM} =$ 2.0\e{-4}, $D_{cGM} =$ 7.0\e{-4}
  and $D_{dGM} =$ 9.0\e{-4}, respectively [mm$^2$s$^{-1}$].
All these values for diffusivity (and the corresponding to the single-fiber response) can be modified by the user with
  custom settings.


\noindent\textbf{\textit{Interpretation and use\textcolon}}
The full simulation workflow, presented in \autoref{fig:figure1}, is implemented using
  \emph{nipype} \citep{gorgolewski_nipype_2011} to ensure reproducibility and usability.
The recommended use case of the simulation workflow is its integration in assessment frameworks
  (\autoref{fig:figure2}, box A) for the evaluation of algorithms and/or pipelines.


To illustrate the features of the \emph{diffantom}, the example dataset underwent a simplified
  connectivity pipeline including \gls*{csd} and probabilistic tractography from
  \emph{MRTrix} \citep{tournier_mrtrix_2012}.
\Gls*{csd} was configured with up to 8 spherical harmonics, and tractography with 1.6$\times$10$^\text{6}$
  seed points evenly distributed across a dilated mask of the \gls*{wm} tissue.
\autoref{fig:figure2}, box B1 shows the result of the tractography obtained with such pipeline.
Finally, we applied \emph{tract querier} \citep{wassermann_on_2013} on segmenting important fiber bundles such
  as the \gls*{cst} and the forceps minor (see \autoref{fig:figure2}, box B2).
Particularly, due to its location nearby the orbitofrontal lobe, the forceps minor is generally affected by
  susceptibility distortions.
The study of this phenomena \citep{esteban_simulationbased_2014} led us to design \emph{diffantom}.


\section*{Discussion and conclusion}

\noindent\textbf{\textit{Discussion\textcolon}}\label{sec:discussion} %
Whole-brain, realistic \gls*{dmri} phantoms are necessary in the developing field of structural
  connectomics.
The \emph{diffantom} is a derivative of \citep{wilkins_fiber_2014} in terms of methodology for
  simulation with two major improvements: the correctness of the \emph{minimally preprocessed} data
  \citep{glasser_minimal_2013} released within the \gls*{hcp}, and a more realistic diffusion
  model.
A possible competitor to \emph{diffantom} is the ground-truth data generated by the organizers of the
  Tractography Challenge held in the ISMRM 2015\footnote{\url{http://www.tractometer.org/ismrm_2015_challenge/}}.
Similarly to \emph{diffantom}, one \gls*{hcp} dataset is used as source of structural information.
However, in this case 25 fiber bundles were manually segmented to obtain a summary tractogram.
Then, making use of \emph{FiberFox}, the segmentation of each bundle is mapped to an analytical description
  which is fed into \emph{Phantomas}.
While this phantom is designed for the bundle-wise evaluation of tractography (with scores such as geometrical coverage,
  valid connections, invalid connections, missed connections, etc. \citep{cote_tractometer_2013}),
  \emph{diffantom} is intended for the connectome-wise evaluation of results, yielding tractography with
  a large number of bundles.
Therefore, \emph{diffantom} and \emph{FiberFox} are complementary as the hypotheses that can be investigated are different.
Moreover, \emph{diffantom} does not require costly manual segmentation of bundles, highly demanding in terms of physiology
  expertise and operation time.
This facilitates the use of \emph{diffantom} as a factory of whole-brain diffusion phantoms.
Lastly, since the gradient scheme can be set by the user, \emph{diffantom} can be seen as a mean to translate the so-called
  \emph{b-matrix} of the \gls*{hcp} to any target scheme.

\noindent\textbf{\textit{Conclusion\textcolon}}\label{sec:conclusion} %
\emph{Diffantom} is a digital phantom generated from a dataset belonging to the \acrlong*{hcp}, and the software factory to
  reproduce it.
The first \emph{diffantom} is presented here to be openly and freely distributed along with the necessary resources to
  generate new \emph{diffantoms}.
We encourage the neuroimage community to contribute with their own \emph{diffantoms} and share them openly.


\section*{Data Sharing}
The first \emph{diffantom} is available under the \gls*{cc0} in {\color{red} (dryad?)}.
The associated software to ``\emph{diffantomize}'' real \gls*{dmri} data is available at \url{https://github.com/oesteban/diffantom}
  under an MIT license.
\emph{Phantomas} is available in \url{https://github.com/ecaruyer/Phantomas} under the revised-BSD license.

\section*{Disclosure/Conflict-of-Interest Statement}

The authors declare that the research was conducted in the absence of any commercial or financial relationships that could be construed as a potential conflict of interest.

\section*{Author Contributions}
All the authors contributed to this study.
OE designed the data generation procedure, implemented the processing pipelines and generated the example dataset.
EC implemented \emph{Phantomas} \citep{caruyer_Phantomas_2014}, helped integrate the project with the simulation routines.
OE, EC, AD thoroughly discussed and framed the aptness of the data in the community.
AD, MBC, MJLC, and AS interpreted the resulting datasets.
MBC, MJLC, and AS advised on all aspects of the study.


\section*{Acknowledgments}
\textit{Funding\textcolon}
This study was supported by the Spanish Ministry of Science and Innovation
  (projects TEC-2013-48251-C2-2-R and INNPACTO XIORT), Comunidad de Madrid (TOPUS) and
  European Regional Development Funds, the Center for Biomedical Imaging
  (CIBM) of the Geneva and Lausanne Universities and the EPFL, as well as the
  Leenaards and Louis Jeantet Foundations.

\nolinenumbers
\bibliographystyle{frontiers/frontiersinSCNS_ENG_HUMS}
%\bibliographystyle{frontiersinHLTH&FPHY} % for Health and Physics articles
\bibliography{Remote}

%%% Upload the *bib file along with the *tex file and PDF on submission if the bibliography is not in the main *tex file

\glsresetall
\linenumbers
\section*{Appendix}\label{sec:appendix}
Let $\{T'_{1\,\cdots\,5}\}$ be the set of original fractions maps obtained with the combined use of FAST and FIRST, FA the
  \gls*{fa} map obtained from the original \gls*{dmri} data, and $\{V_{1\,\cdots\,3}, F'_{1\,\cdots\,3}\}$ the direction
  maps $V_i$ of restricted diffusion by fibers with their estimated volume fractions $F'_i$ calculated with BEDPOSTX.
The final $\{T_{1\,\cdots\,5}\}$ maps of isotropic fractions are computed as follows:

  \begin{align*}
  T_1 &= (1.0-w_{cgm}) \cdot T'_1 \\
  T_2 &= (1.0-w_{dgm}) \cdot T'_2 \\
  T_3 &= (1.0-w_{wm}) \cdot T'_3 \\
  T_4 &= T'_4 \\
  T_5 &= 0.0
  \end{align*}
where $w_{\{cgm, dgm, wm\}}$ are weighing factors of hindered diffusion for each tissue
  (defaults are $w_{cgm} =$ 25\%, $w_{dgm} =$ 50\%, $w_{cgm} =$ 95\%).
The final $\{F_{1\,\cdots\,3}\}$ maps are computed as follows:

\begin{align*}
F_1 &= w_{wm} \cdot T_2 \cdot \text{FA} + w_{f1} (w_{cgm} \cdot T_1 + w_{dgm} \cdot T_2) \\
F_2 &= w_{wm} \cdot T_2 - (F_1 + F_3) + w_{f2} (w_{cgm} \cdot T_1 + w_{dgm} \cdot T_2) \\
F_3 &= w_{wm} \cdot F'_3 + w_{f3} (w_{cgm} \cdot T_1 + w_{dgm} \cdot T_2)
\end{align*}
where $w_{\{f1, f2, f3\}}$ are the contributions of the \gls*{gm} compartments to each fiber population.
By default: $w_{f1} = $ 48\%, $w_{f2} = $ 37\%, $w_{f3} = $ 15\%.
Finally, the resulting maps are normalized to fulfill $\sum_i T_i + \sum_i F_i = 1.0$.

\newpage
\section*{Figures}

\begin{figure}[h!]
\begin{center}
% \includegraphics[width=\linewidth]{figures/figure01A}
% \includegraphics[width=\linewidth]{figures/figure01B}
\includestandalone[width=\linewidth]{figure01}
\end{center}
\textbf{\refstepcounter{figure}\label{fig:figure1} Figure \arabic{figure}. }%
{\textbf{Generation of \emph{diffantoms}:} First, a microstructural model is generated as described
  in the \nameref{sec:data_generation} subsection.
Please note the piece-wise linear function of the color scale to enable visibility of small volume fractions.
This microstructural model is fed into a simulation block that generates the final \gls*{dmri} dataset
  as described in the second block.
}
\end{figure}

\begin{figure}[h!]
\begin{center}
\includegraphics[width=\linewidth]{figures/figure02}
\end{center}
\textbf{\refstepcounter{figure}\label{fig:figure2} Figure \arabic{figure}. }%
{\textbf{A. Recommended use of \emph{diffantom}}.
The dataset can be used to test algorithms and pipelines against a perturbation that is
  introduced in \emph{diffantom}.
Perturbations can model typical artifacts found in \gls*{dmri} datasets or pathological conditions
  such as a tumor.
\textbf{B1. Tractogram of fibers crossing slice 56 of the \emph{diffantom}} represented over that corresponding
  slice of the \emph{b0} volume. \textbf{B2. Segmentation of some fiber bundles}: the right \gls*{cst} is
  represented in blue color, the left \gls*{cst} in red, and the forceps minor in green.
  The box includes the slice 56 of the \emph{b0} and the pial surface is represented with a light gray shadow.
}
\end{figure}

\end{document}
