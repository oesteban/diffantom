%%%%%%%%%%%%%%%%%%%%%%%%%%%%%%%%%%%%%%%%%%%%%%%%%%%%%%%%%%%%%%%%%%%%%%%%%%%%%%%%%%%%%%%%%%%%%%%%%%%%%%%%%%%%%%%%%%%%%%%%%%%%%%%%%%%%%%%%%%%%%%%%%%%%%%%%%%%
% This is just an example/guide for you to refer to when submitting manuscripts to Frontiers, it is not mandatory to use Frontiers .cls files nor frontiers.tex  %
% This will only generate the Manuscript, the final article will be typeset by Frontiers after acceptance.                                                 %
%                                                                                                                                                         %
% When submitting your files, remember to upload this *tex file, the pdf generated with it, the *bib file (if bibliography is not within the *tex) and all the figures.
%%%%%%%%%%%%%%%%%%%%%%%%%%%%%%%%%%%%%%%%%%%%%%%%%%%%%%%%%%%%%%%%%%%%%%%%%%%%%%%%%%%%%%%%%%%%%%%%%%%%%%%%%%%%%%%%%%%%%%%%%%%%%%%%%%%%%%%%%%%%%%%%%%%%%%%%%%%

%%% Version 3.1 Generated 2015/22/05 %%%
%%% You will need to have the following packages installed: datetime, fmtcount, etoolbox, fcprefix, which are normally inlcuded in WinEdt. %%%
%%% In http://www.ctan.org/ you can find the packages and how to install them, if necessary. %%%

\documentclass[english]{frontiers/frontiersSCNS} % for Science, Engineering and Humanities and Social Sciences articles
\usepackage{xifthen}
\newboolean{review}
\setboolean{review}{true}

\usepackage[mode=buildnew]{standalone}
\usepackage{tikz}
\usepackage[framemethod=TikZ]{mdframed}
%\setcitestyle{square}
\usepackage{url,microtype}
\usepackage[right]{lineno}
\usepackage[toc,nomain,acronym,shortcuts,translate=false]{glossaries}
\usepackage[colorlinks=true,linkcolor=black, citecolor=black!80, urlcolor=black!80]{hyperref}
\usepackage{doi}
\usepackage[onehalfspacing]{setspace}


\ifthenelse{\boolean{review}}
{
  \usepackage[colorinlistoftodos, textwidth=1.4cm]{todonotes}
  \setlength{\marginparwidth}{1.6cm}
  \reversemarginpar
  \newcommand{\revcomment}[2][]{\ifthenelse{\equal{#1}{}}{}{\todo[size=\tiny]{#1}}%
  \begingroup \color{blue!70} #2\endgroup}
}
{
  \newcommand{\todo}[2][]{}
  \newcommand{\revcomment}[2][]{#2}
}

% \usepackage{showframe}
\def\keyFont{\fontsize{7}{9}\helveticabold }
\def\firstAuthorLast{Esteban {et~al.}} %use et al only if is more than 1 author
\def\Authors{Oscar Esteban\,$^{1,2,*}$, Emmanuel Caruyer\,$^{3}$, Alessandro Daducci\,$^{4}$, Meritxell Bach-Cuadra\,$^{4,5}$,%
Mar\'ia-J. Ledesma-Carbayo\,$^{1,2}$ and Andres Santos\,$^{1,2}$}

\def\Address{%
$^{1}$Biomedical Image Technologies (BIT), ETSI Telecom., Universidad Polit\'ecnica de Madrid, Madrid, Spain \\
$^{2}$Centro de Investigaci\'on Biom\'edica en Red en Bioingenier\'ia, Biomateriales y Nanomedicina (CIBER-BBN), Spain \\
$^{3}$CNRS, IRISA (UMR 6074), VisAGeS research group, Rennes, France \\
$^{4}$Signal Processing Laboratory (LTS5), \'Ecole Polytechnique F\'ed\'erale de Lausanne (EPFL), Lausanne, Switzerland \\
${^5}$Dept. of Radiology, CIBM, University Hospital Center (CHUV) and University of Lausanne (UNIL), Lausanne, Switzerland %
}
% The Corresponding Author should be marked with an asterisk
% Provide the exact contact address (this time including street name and city zip code) and email of the corresponding author
\def\corrAuthor{Oscar Esteban}
\def\corrAddress{Biomedical Image Technologies (BIT), ETSI Telecomunicaci\'on, Av. Complutense 30, C203, E28040 Madrid, Spain}
\def\corrEmail{phd@oscaresteban.es}

% -*- root: main.tex -*-
% @Author: Oscar Esteban
% @Date:   2015-06-16 15:45:32
% @Last Modified by:   Oscar Esteban
% @Last Modified time: 2015-09-15 16:42:48

\newacronym{dmri}{dMRI}{diffusion MRI}
\newacronym{fa}{FA}{fractional anisotropy}
\newacronym{adc}{ADC}{anisotropic diffusion coefficient}
% \newacronym{hcp}{HCP}{Human Connectome Project}
\newacronym{fod}{FOD}{fiber orientation distribution}
\newacronym{fodf}{fODF}{fiber orientation distribution function}
\newacronym{wm}{WM}{white matter}
\newacronym{gm}{GM}{gray matter}
\newacronym{cgm}{cGM}{cortical gray matter}
\newacronym{dgm}{dGM}{deep gray matter}
\newacronym{csf}{CSF}{cerebrospinal fluid}
\newacronym{snr}{SNR}{signal-to-noise ratio}
\newacronym{csd}{CSD}{constrained spherical deconvolution}
\newacronym{cc0}{CC0}{Creative Commons Zero licence}
\newacronym{roc}{ROC}{receiver operating characteristic}
\newacronym{pve}{PVE}{partial volume effect}
\newacronym{cst}{CST}{corticospinal tract}
\newacronym{t1}{T1w}{T1-weighted}

\newglossaryentry{hcp}
{
	name={Human Connectome Project},
	text={HCP},
	first={Human Connectome Project (HCP, \cite{essen_human_2012})},
	long={Human Connectome Project},
	description={Human Connectome Project}
}

\newglossaryentry{bids}%
{
	name={BIDS},
	first={BIDS (Brain Imaging Data Structure, \cite{gorgolewski_brain_2015})},
	long={Brain Imaging Data Structure},
	description={Brain Imaging Data Structure}
}

\newglossaryentry{bedpostx}%
{
	name={BEDPOSTX},
	first={BEDPOSTX (Bayesian Estimation of Diffusion Parameters Obtained using Sampling Techniques modelling crossing --X-- fibres, \cite{jbabdi_modelbased_2012})},
	description={Bayesian Estimation of Diffusion Parameters Obtained using Sampling Techniques modelling crossing --X-- fibres}
}
\newacronym[first={FAST (FMRIB's Automated Segmentation Tool, \cite{zhang_segmentation_2001})}]%
{fast}{FAST}{FMRIB's Automated Segmentation Tool}
\newacronym[first={FIRST (FMRIB's Integrated Registration and Segmentation Tool, \cite{patenaude_bayesian_2011})}]%
{first}{FIRST}{FMRIB's Integrated Registration and Segmentation Tool}

\makeglossaries



\providecommand{\diffantom}{\emph{Diffantom}}
\providecommand{\Diffantom}{\emph{Diffantom}}
\newcommand{\lowb}{\textit{b0}}
\newcommand{\e}[1]{\ensuremath{\;\cdot\,\text{10}^\text{#1}}}
\newcommand{\vmaps}{\ensuremath{\{\mathbf{V}_i \,|\, i \in \{1,2,3\}\}}}
\newcommand{\fmaps}{\ensuremath{\{F_i \,|\, i \in \{1,2,3\}\}}}
\newcommand{\tmaps}{\ensuremath{\{T_j \,|\, j \in \{1,\ldots,5\}\}}}
\newcommand{\diffunits}{\ensuremath{\text{mm}^2\text{s}^{\text{-1}}}}

\begin{document}

\onecolumn
\firstpage{1}

\title[Diffantom]{Diffantom: whole-brain diffusion MRI phantoms derived from real datasets of the \acrlong{hcp}}

\author[\firstAuthorLast ]{\Authors} %This field will be automatically populated
\address{} %This field will be automatically populated
\correspondance{} %This field will be automatically populated

\extraAuth{}% If there are more than 1 corresponding author, comment this line and uncomment the next one.
%\extraAuth{corresponding Author2 \\ Laboratory X2, Institute X2, Department X2, Organization X2, Street X2, City X2 , State XX2 (only USA, Canada and Australia), Zip Code2, X2 Country X2, email2@uni2.edu}


\maketitle
\linenumbers

% \begin{mdframed}[hidealllines=true, backgroundcolor=black!5, roundcorner=5pt]
\section*{Diffantom in brief}
\revcomment[Rev.\#2\\Q.10]{%
\diffantom{} is a whole-brain \gls*{dmri} phantom publicly available
  through the Dryad Digital Repository (\doi{10.5061/dryad.4p080}).
The dataset contains two single-shell \gls*{dmri} images, along with the corresponding gradient
  information, packed following the \gls*{bids}.
The released dataset is designed for the evaluation of the impact of susceptibility distortions
  and benchmarking existing correction methods.
}%

\revcomment[Rev.\#1\\Q14.1]{%
In this Data Report we also release the software instruments involved in generating \emph{diffantoms},
  so that researchers are able to generate new phantoms derived from different subjects,
  and apply these data in other applications like investigating diffusion sampling
  schemes, the assessment of \gls*{dmri} processing methods, the simulation of pathologies
  and imaging artifacts, etc.
In summary, \diffantom{} is intended for unit testing of novel methods, cross-comparison of established methods,
  and integration testing of partial or complete processing flows to extract connectivity networks
  from \gls{dmri}.}

\section*{Introduction}
Fiber tracking on \gls*{dmri} data has become an important tool for the \textit{in-vivo} investigation
  of the structural configuration of fiber bundles at the macroscale.
Tractography is fundamental to gain information about \gls*{wm} morphology in many clinical applications
  like neurosurgical planning \citep{golby_interactive_2011}, post-surgery evaluations \citep{toda_utility_2014},
  and the study of neurological diseases as in \citep{chua_diffusion_2008} addressing multiple sclerosis and
  Alzheimer's disease.
The analysis of structural brain networks using graph theory is also applied on tractography,
  for instance in the definition of the unique subject-wise patterns of connectivity
  \citep{sporns_human_2005}, in the assessment of neurological diseases \citep{griffa_structural_2013}, and in the
  study of the link between structural and functional connectivity \citep{messe_predicting_2015}.
However, the development of the field is limited by the lack of a gold standard to test and compare the
  wide range of methodologies available for processing and analyzing \gls*{dmri}.

Large efforts have been devoted to the development of physical phantoms
  \citep{lin_validation_2001,campbell_flowbased_2005,perrin_validation_2005,fieremans_simulation_2008,tournier_resolving_2008}.
\cite{cote_tractometer_2013} conducted a thorough review of tractography methodologies using the
  so-called \emph{FiberCup} phantom \citep{poupon_new_2008,fillard_quantitative_2011}.
These phantoms are appropriate to evaluate the angular resolution in fiber crossings and accuracy of
  direction-independent scalar parameters in very simplistic geometries.
Digital simulations are increasingly popular because the complexity of whole-brain tractography
  can not be accounted for with current materials and proposed methodologies to build physical phantoms.
Early digital phantoms started with simulation of simple geometries
  \citep{basser_in_2000,goessl_fiber_2002,tournier_limitations_2002,leemans_mathematical_2005}
  to evaluate the angular resolution as well.
These tools generally implemented the multi-tensor model \citep{alexander_analysis_2001,tuch_high_2002}
  to simulate fiber crossing, fanning, kissing, etc.
\cite{close_software_2009} presented the \emph{Numerical Fibre Generator}, a software to simulate
  spherical shapes filled with digital fiber tracts.
\cite{caruyer_phantomas_2014} proposed \emph{Phantomas} to simulate any kind of analytic geometry
  inside a sphere.
\emph{Phantomas} models diffusion by a restricted and a hindered compartment, similar to
  \citep{assaf_composite_2005}.
\cite{wilkins_fiber_2015} proposed a whole-brain simulated phantom derived from voxel-wise orientation
  of fibers averaged from real \gls*{dmri} scans and the multi-tensor model with a compartment of
  isotropic diffusion.
\cite{neher_fiberfox_2014} proposed \emph{FiberFox}, a visualization software to develop
  complex geometries and their analytical description.
Once the geometries are obtained, the software generates the corresponding \gls*{dmri} signal with a
  methodology very close to that implemented in \emph{Phantomas}.
An interesting outcome of \emph{FiberFox} is the phantom dataset\footnote{Available at
  \url{http://www.tractometer.org/ismrm_2015_challenge/}} created for the Tractography
  Challenge held in ISMRM 2015.
This dataset was derived from the tractography extracted in one \gls*{hcp} dataset.
In the tractogram, 25 fiber bundles of interest were manually segmented by experts.
Using \emph{FiberFox}, the segmentation of each bundle was mapped to an analytical
  description, and finally simulated the signal.

In this data report we present \diffantom{}, an \emph{in-silico} dataset to assess tractography and connectivity
  pipelines using \gls*{dmri} real data as source microstructural information.
\Diffantom{} is inspired by the work of \cite{wilkins_fiber_2015}, with two principal novelties.
First, since we use a dataset from the \gls*{hcp} as input, data are already corrected for the most
  relevant distortions.
The second improvement is a more advanced signal model to generate the phantom using the hindered and restricted
  diffusion model of \emph{Phantomas} \citep{caruyer_phantomas_2014}.
As a result, we provide a whole-brain digital phantom of \gls*{dmri} data with structural information derived
  from an \gls*{hcp} dataset.
We also openly release the \emph{diffantomizer} workflow, the software package necessary to generate custom \emph{diffantoms}.
\Diffantom{} is originally designed for the investigation of susceptibility-derived distortions, a
  typical artifact that produces geometrical warping in certain regions of \gls*{dmri} datasets.
In \citep{esteban_simulationbased_2014} we addressed this phenomenon and concluded that the connectivity
  matrix of \emph{Phantomas} was not dense enough to evaluate the integration of correction methods
  in pipelines for the connectome extraction.

\section*{Data description}

\noindent\textbf{\textit{Microstructural model\textcolon}\label{sec:diffantom-data-micromodel}} %
The simulation process relies on a microstructural model derived from real data.
On one hand, the \emph{diffantomizer} workflow requires up to five fraction maps \tmaps{} of
  free- and hindered- diffusion (see \autoref{fig:diffantom-01}, panel A).
These compartments will be derived from the macroscopic structure of tissues within the brain,
  specified in the following order\footnote{Corresponding to the \emph{5TT format} established
  with the latest version 3.0 of \emph{MRTrix} \citep{tournier_mrtrix_2012}}:
  \gls*{cgm}, \gls*{dgm}, \gls*{wm}, \gls*{csf}, and abnormal tissue\footnote{Since here we
  simulate healthy subjects, the last fraction map $T_5$ is empty and can be omitted}.
On the other hand, the restricted-diffusion compartments are specified by up to three volume fractions \fmaps{}
  of three single fiber populations per voxel along with their corresponding direction maps \vmaps{}.

The process to obtain the microstructural model from one dataset of the \gls*{hcp} can be described
  as follows (see also \autoref{fig:diffantom-01}, panel B):
1) The fiber orientation maps $\{\mathbf{V}_i\}$ and their corresponding estimations of volume fraction $\{F'_i\}$ are
  obtained using the ball-and-stick model for multi-shell data of \gls*{bedpostx}
  on the \gls*{dmri} data.
The \gls*{hcp} recommends \gls*{bedpostx} to reconstruct their data \citep{glasser_minimal_2013}.
A further advantage is that \gls*{bedpostx} exploits the multi-shell acquisitions of the \gls*{hcp} while
  operating at whole-brain level.
2) A \gls*{fa} map is obtained after fitting a tensor model with \emph{MRTrix}.
As we shall see in the Appendix, the \gls*{fa} is used to infer $F_1$ (the fraction map of the most prevalent fiber),
  avoiding the extremely noisy estimation of $F'_1$ performed by \gls*{bedpostx} in the previous step.
3) The original fiber fractions $\{F'_i\}$ and the \gls*{fa} map are denoised with a nonlocal means filter included
  in \emph{dipy} \citep{garyfallidis_dipy_2014}.
This step produces an important smoothing of the maps, while preserving the edges.
Smoothing is also beneficial in simplifying the voxel-wise diffusion model.
4) The macrostructural fractions $\{T'_j\}$ are extracted from the \acrlong*{t1} image of the dataset,
  using standard \emph{FSL} segmentation tools \citep{jenkinson_fsl_2012}.
5) The images obtained previously (\gls*{fa} map, $\{\mathbf{V}_i\}$, $\{F'_i\}$, and $\{T'_j\}$)
  are combined as described in the \nameref{sec:appendix} to generate the final microstructural model
  ($\{\mathbf{V}_i\}$, $\{F_i\}$, and $\{T_j\}$), presented in \autoref{fig:diffantom-01}-A.

\noindent\textbf{\textit{Diffusion signal generation\textcolon}\label{sec:diffantom-data-dwi}} %
Once a microstructural model of the subject has been synthesized, the fiber orientation maps $\{\mathbf{V}_i\}$
  are weighted by the fiber-fraction maps $\{F_i\}$ and projected onto a continuous representation of
  the \glspl{fod}.
A close-up showing how the \glspl{fod} map looks is presented in \autoref{fig:diffantom-01}B.
The single fiber response is a Gaussian diffusion tensor with axial symmetry and eigenvalues $\lambda_1=$ 2.2\e{-3}
  \diffunits{} and $\lambda_{2,3}=$ 0.2\e{-3} \diffunits{}.
The resulting \glspl{fod} map is then combined with the free- and hindered-diffusion compartments corresponding to $\{T_j\}$.
The free-diffusion compartment corresponds to the \gls*{csf} fraction map $T_4$ and is modeled with isotropic
  diffusivity $D_{CSF}$ of 3.0\e{-3} \diffunits{}.
The hindered-diffusion compartments correspond to $\{T_1,T_2,T_3\}$ and are also modeled with isotropic diffusivity
  $D_{WM} =$ 2.0\e{-4}, $D_{cGM} =$ 7.0\e{-4} and $D_{dGM} =$ 9.0\e{-4}, respectively [\diffunits{}].
All these values for diffusivity (and the corresponding to the single-fiber response) can be modified by the user with
  custom settings.
The restricted- and hindered- compartments are then fed into \emph{Phantomas} \citep{caruyer_phantomas_2014}
  and the final \gls*{dmri} signal is obtained.
By default, diffusion data are generated using a scheme of 100 directions distributed in one shell with uniform
  coverage \citep{caruyer_design_2013}.
Custom one- or multi-shell schemes can be generated supplying the tables of corresponding vectors and $b$-values.
Rician noise is also included in \emph{Phantomas}, and the \gls*{snr} can be set by the user.
The default value for \gls*{snr} is preset to 30.0.


\noindent\textbf{\textit{Implementation and reproducibility\textcolon}\label{sec:data_workflow}} %
We also provide the \emph{diffantomizer} workflow, the software package used to generate \emph{diffantoms}, so
  that users can regenerate similar datasets with different parameters.
This workflow, presented in \autoref{fig:diffantom-01}, is implemented using
  \emph{nipype} \citep{gorgolewski_nipype_2011} to ensure reproducibility and usability.

\noindent\textbf{\textit{Interpretation and recommended uses\textcolon}\label{sec:diffantom-data-use}} %
To illustrate the features of \diffantom{}, the example dataset underwent a simplified
  connectivity pipeline including \gls*{csd} and probabilistic tractography from
  \emph{MRTrix} \citep{tournier_mrtrix_2012}.
\Gls*{csd} was reconstructed using 8$^\text{th}$-order spherical harmonics, and tractography with 1.6\e{6}
  seed points evenly distributed across a dilated mask of the \gls*{wm} tissue.
\autoref{fig:diffantom-02}, panels A1 and A3, show the result of the tractography obtained with such pipeline for
  the original \diffantom{} and a distorted version.
Finally, we applied \emph{tract querier} \citep{wassermann_on_2013} to segment some fiber bundles such
  as the \gls*{cst} and the forceps minor (see \autoref{fig:diffantom-02}, panels A2, A4).
Particularly, due to its location nearby the orbitofrontal lobe, the forceps minor is generally affected by
  susceptibility distortions.

We recommend \diffantom{} as ground-truth in verification and validation frameworks
  (\autoref{fig:diffantom-02}, panel B) for testing pipelines.
\Diffantom{} is applicable in the unit testing of algorithms, the integration testing of
  modules in workflows, and the overall system testing.
Some potential applications follow:
\begin{itemize}
  \item Investigating the impact of different diffusion sampling schemes on the local microstructure
    model of choice and on the subsequent global tractography outcome.
  Since the gradient scheme can be set by the user, \diffantom{} can be seen as a mean to translate the so-called
  \emph{b-matrix} of the source dataset to any target scheme.
  \item Assessment of sensitivity and robustness to imaging artifacts (noise, \acrlong{pve} and \gls*{csf} contamination,
    susceptibility-derived warping, Eddy-currents-derived distortions, etc.) at unit, integration and systems testing levels.
  \item Using \diffantom{} as in panel B of \autoref{fig:diffantom-02}, it is possible to apply binary classification measures
    to evaluate the resulting connectivity matrix.
  Considering the connectivity matrix of the \emph{reference Diffantom} and the resulting
    matrix of the \emph{test Diffantom}, the \acrfull*{roc} of the pipeline can be characterized.
  \item Simulation of pathological brains by altering the microstructural
    model accordingly \cite[e.g. as tumors were simulated in][]{kaus_simulation_2000}.
\end{itemize}
In order to exemplify one of these intended uses, we also release a \diffantom{} including the susceptibility-derived
  distortion in simulation.
\revcomment[Rev.\#1\\Q14.2]{%
These two images belong to a broader dataset, automatically generated, used in a study to quantify
  the impact of susceptibility distortions and correction methods on the connectome extraction
  \cite[Chapter 5]{esteban_image_2015}.
In this study, three widely-used correction methods are compared in a reference framework of
  several \emph{Diffantoms} with realistic and controlled distortions.
This context provides a useful resource to characterize the impact of susceptibility distortion
  on the final connectivity network and allows the evaluation of the different correction
  methodologies available.%
}

\section*{Discussion and conclusion}
\noindent\textbf{\textit{Discussion\textcolon}}\label{sec:discussion} %
Whole-brain, realistic \gls*{dmri} phantoms are necessary in the developing field of structural
  connectomics.
\Diffantom{} is a derivative of \citep{wilkins_fiber_2015} in terms of methodology for
  simulation with two major advances.
First, the correctness of the \emph{minimally preprocessed} data \citep{glasser_minimal_2013}
  released within the \gls*{hcp}.
\cite{wilkins_fiber_2015} explicitly state that their original data were not corrected for certain artifacts,
  and thus, generated data are affected correspondingly.
Second, \diffantom{} implements the hindered and restricted compartments model \citep{assaf_composite_2005},
  which is a more complete model than the multi-tensor diffusion model.

A possible competitor to \diffantom{} is the phantom generated for the Tractography Challenge in
  ISMRM 2015.
Similarly to \diffantom{}, the organizers used an \gls*{hcp} subject as source of structural information.
While this phantom is designed for the bundle-wise evaluation of tractography (with the scores defined in the
  \emph{Tractometer} \citep{cote_tractometer_2013}, such as geometrical coverage, valid connections, invalid connections,
  missed connections, etc.), \diffantom{} is intended for the connectome-wise evaluation of results,
  yielding a tractography with a large number of bundles.
Therefore, \diffantom{} and \emph{FiberFox} are complementary as the hypotheses that can be investigated are different.
Moreover, \diffantom{} does not require costly manual segmentation of bundles, highly demanding in terms of physiology
  expertise and operation time.
The software workflow released with this data report (the \emph{diffantomizer}) ensures the reproducibility of
  \diffantom{} and enables the generation of custom \emph{diffantoms}.
The \emph{diffantomizer} is designed for, but not limited to, use \gls*{hcp} datasets as source of structural information.

\noindent\textbf{\textit{Conclusion\textcolon}}\label{sec:conclusion} %
\Diffantom{} is a whole-brain digital phantom generated from a dataset from the \acrlong*{hcp}.
\Diffantom{} is presented here to be openly and freely distributed along with the \emph{diffantomizer} workflow
  to generate new \emph{diffantoms}.
We encourage the neuroimage community to contribute with their own \emph{diffantoms} and share them openly.


\section*{Data Sharing}
The first \diffantom{} and its distorted version are available under the \gls*{cc0} using the Dryad Digital Repository
  (\doi{10.5061/dryad.4p080}).
The package is organized following the \gls*{bids} standard.
The associated software to ``\emph{diffantomize}'' real \gls*{dmri} datasets is available at
  \url{https://github.com/oesteban/diffantom} under an MIT license.
\emph{Phantomas} is available in \url{https://github.com/ecaruyer/Phantomas} under the revised-BSD license.

\section*{Disclosure}

The authors declare that the research was conducted in the absence of any commercial or financial relationships that could be construed as a potential conflict of interest.

\section*{Author Contributions}
All the authors contributed to this study.
OE designed the data generation procedure, implemented the processing pipelines and generated the example dataset.
EC implemented \emph{Phantomas} \citep{caruyer_phantomas_2014}, helped integrate the project with the simulation routines.
OE, EC, AD thoroughly discussed and framed the aptness of the data in the community.
AD, MBC, MJLC, and AS interpreted the resulting datasets.
MBC, MJLC, and AS advised on all aspects of the study.


\section*{Acknowledgments}
We thank Gert Wollny for his revision of this work.
\textit{Funding\textcolon}
This study was supported by the Spanish Ministry of Science and Innovation
  (projects TEC-2013-48251-C2-2-R and INNPACTO XIORT), Comunidad de Madrid (TOPUS) and
  European Regional Development Funds, the Center for Biomedical Imaging
  (CIBM) of the Geneva and Lausanne Universities and the EPFL, as well as the
  Leenaards and Louis Jeantet Foundations.

\nolinenumbers
\newpage
\bibliographystyle{frontiers/frontiersinSCNS_ENG_HUMS}
%\bibliographystyle{frontiersinHLTH&FPHY} % for Health and Physics articles
\bibliography{Remote}
\newpage

%%% Upload the *bib file along with the *tex file and PDF on submission if the bibliography is not in the main *tex file

\glsresetall
\linenumbers
\section*{Appendix}\label{sec:appendix}
Let $\{T'_j\}$ be the set of original fractions maps obtained with \path{act_anat_prepare_fsl}, a
  tool in \emph{MRTrix} that combines \gls*{fast} and \gls*{first}
  to generate the macrostructural 5TT map.
FA denotes the \gls*{fa} map obtained from the original \gls*{dmri} data: the local fiber orientation maps
  $\{\mathbf{V}_i\}$ with their estimated volume fractions $\{F'_i\}$ calculated with \gls*{bedpostx}.
The final $\{T_j\}$ maps of isotropic fractions are computed as follows:

  \begin{align*}
  T_1 &= (1.0-f_{cgm}) \cdot T'_1 \\
  T_2 &= (1.0-f_{dgm}) \cdot T'_2 \\
  T_3 &= (1.0-f_{wm}) \cdot T'_3 \\
  T_4 &= T'_4 \\
  T_5 &= 0.0
  \end{align*}
where $f_{\{cgm, dgm, wm\}}$ are the fractions of restricted diffusion for each tissue.
\cite{sepehrband_brain_2015} found out that the fiber fraction ranges across the corpus
  callosum from the 70$\pm$8\% in its body to an upper bound of 80$\pm$11\% in the splenium.
Therefore, we choose $f_{wm} =$ 80\% as default fraction of restricted diffusion in the
  \gls*{wm}.
To our knowledge, restricted diffusion fractions have been studied only for \gls*{wm}.
Therefore, we set $f_{cgm} =$ 25\% and $f_{dgm} =$ 50\% as they yield plausible \gls*{fa}
  and \gls*{adc} maps, assessed visually.
The final $\{F_i\}$ maps are computed as follows:

\begin{align*}
F_1 &= f_{wm} \cdot T_2 \cdot \text{FA} + w_{f1} (f_{cgm} \cdot T_1 + f_{dgm} \cdot T_2) \\
F_2 &= f_{wm} \cdot T_2 - (F_1 + F_3) + w_{f2} (f_{cgm} \cdot T_1 + f_{dgm} \cdot T_2) \\
F_3 &= f_{wm} \cdot F'_3 + w_{f3} (f_{cgm} \cdot T_1 + f_{dgm} \cdot T_2)
\end{align*}
where $w_{\{f1, f2, f3\}}$ are the contributions of the \gls*{gm} compartments to each fiber population.
By default: $w_{f1} = $ 48\%, $w_{f2} = $ 37\%, $w_{f3} = $ 15\%.
Finally, the resulting maps are normalized to fulfill $\sum_j T_j + \sum_i F_i = 1.0$.

\newpage
\section*{Figures}

\begin{figure}[h!]
\begin{center}
\includestandalone[width=\linewidth]{figure01}
\end{center}
\textbf{\refstepcounter{figure}\label{fig:diffantom-01} Figure \arabic{figure}. }{%
\textbf{A. Microstructural model of \diffantom{}}.
The phantom is simulated from an underlying microstructural model specified with the
  following volume-fraction maps: three hindered-diffusion compartments $\{T_1, T_2, T_3\}$,
  one free-diffusion compartment $T_4$ corresponding to the \gls*{csf},
  three restricted-diffusion compartments $\{F_i\}$, and three vectorial maps
  associated with the local fiber directions $\{\mathbf{V}_i\}$.
Please note the piece-wise linear function of the color scale to enable visibility of small volume fractions.
\textbf{B. The \emph{diffantomizer} workflow, a workflow to generate \emph{diffantoms}}.
The pipeline to generate phantoms from any \gls*{hcp} dataset is presented in the lower panel.
Once the microstructural model shown in the upper panel has been prepared as described in \nameref{sec:diffantom-data-micromodel},
  the local orientations are computed and fed into \emph{Phantomas} to finally simulate the signal.
}
\end{figure}

\begin{figure}[h!]
\begin{center}
\includestandalone[width=\linewidth]{figure02}
\end{center}
\textbf{\refstepcounter{figure}\label{fig:diffantom-02} Figure \arabic{figure}. }{%
\textbf{A. Example dataset.}
A1 and A3 show the tractogram of fibers crossing slice 56 of \diffantom{} as
  extracted with \emph{MRTrix}, represented over the corresponding slice of the
  \lowb{} volume for the original (A1) and the distorted (A3) phantoms, with a gray
  frame highlighting the absence of important tracks.
Panels A2 and A4 show the segmentation of the right \gls*{cst} represented with blue
  streamlines, the left \gls*{cst} (red streamlines), and the forceps minor (green streamlines)
  using \emph{tract\_querier}.
A2 and A4 include the slice 56 of the \lowb{} and the pial surface is represented
  with transparency.
In the distorted \diffantom{} (A4) the forceps minor was not detected.
\textbf{B. Recommended use of \diffantom{}}.
The phantom is designed to be used as ground-truth information in evaluation frameworks,
  to implement unit test of algorithms, to check integration of processing units within
  pipelines or to validate complete workflows.
For instance, in order to evaluate artifacts, a perturbation can be induced in the microstructural
  model or after simulation to provide reference and test datasets.
}
\end{figure}

\end{document}
