%%%%%%%%%%%%%%%%%%%%%%%%%%%%%%%%%%%%%%%%%%%%%%%%%%%%%%%%%%%%%%%%%%%%%%%%%%%%%%%%%%%%%%%%%%%%%%%%%%%%%%%%%%%%%%%%%%%%%%%%%%%%%%%%%%%%%%%%%%%%%%%%%%%%%%%%%%%
% This is just an example/guide for you to refer to when producing your supplementary material for your Frontiers article.                                 %
%%%%%%%%%%%%%%%%%%%%%%%%%%%%%%%%%%%%%%%%%%%%%%%%%%%%%%%%%%%%%%%%%%%%%%%%%%%%%%%%%%%%%%%%%%%%%%%%%%%%%%%%%%%%%%%%%%%%%%%%%%%%%%%%%%%%%%%%%%%%%%%%%%%%%%%%%%%

%%% Version 2.1 Generated 2015/05/22 %%%
%%% You will need to have the following packages installed: datetime, fmtcount, etoolbox, fcprefix, which are normally inlcuded in WinEdt. %%%
%%% In http://www.ctan.org/ you can find the packages and how to install them, if necessary. %%%

\documentclass{frontiers/frontiers_suppmat} % for all articles
\usepackage[toc,nomain,acronym,shortcuts,translate=false]{glossaries}
\usepackage{url,microtype}
\usepackage[colorlinks=true,linkcolor=black, citecolor=black!80, urlcolor=black!80]{hyperref}
\usepackage{doi}
\usepackage[onehalfspacing]{setspace}

% Leave a blank line between paragraphs in stead of using \\


\def\keyFont{\fontsize{8}{11}\helveticabold }

\def\firstAuthorLast{Esteban {et~al.}} %use et al only if is more than 1 author
\def\Authors{Oscar Esteban*, Emmanuel Caruyer, Alessandro Daducci, Meritxell Bach-Cuadra,%
Mar\'ia-J. Ledesma-Carbayo and Andres Santos}

% The Corresponding Author should be marked with an asterisk
% Provide the exact contact address (this time including street name and city zip code) and email of the corresponding author
\def\corrAuthor{Oscar Esteban}
\def\corrAddress{Biomedical Image Technologies (BIT), ETSI Telecomunicaci\'on, Av. Complutense 30, C203, E28040 Madrid, Spain}
\def\corrEmail{phd@oscaresteban.es}


% -*- root: main.tex -*-
% @Author: Oscar Esteban
% @Date:   2015-06-16 15:45:32
% @Last Modified by:   Oscar Esteban
% @Last Modified time: 2015-09-15 16:42:48

\newacronym{dmri}{dMRI}{diffusion MRI}
\newacronym{fa}{FA}{fractional anisotropy}
\newacronym{adc}{ADC}{anisotropic diffusion coefficient}
% \newacronym{hcp}{HCP}{Human Connectome Project}
\newacronym{fod}{FOD}{fiber orientation distribution}
\newacronym{fodf}{fODF}{fiber orientation distribution function}
\newacronym{wm}{WM}{white matter}
\newacronym{gm}{GM}{gray matter}
\newacronym{cgm}{cGM}{cortical gray matter}
\newacronym{dgm}{dGM}{deep gray matter}
\newacronym{csf}{CSF}{cerebrospinal fluid}
\newacronym{snr}{SNR}{signal-to-noise ratio}
\newacronym{csd}{CSD}{constrained spherical deconvolution}
\newacronym{cc0}{CC0}{Creative Commons Zero licence}
\newacronym{roc}{ROC}{receiver operating characteristic}
\newacronym{pve}{PVE}{partial volume effect}
\newacronym{cst}{CST}{corticospinal tract}
\newacronym{t1}{T1w}{T1-weighted}

\newglossaryentry{hcp}
{
	name={Human Connectome Project},
	text={HCP},
	first={Human Connectome Project (HCP, \cite{essen_human_2012})},
	long={Human Connectome Project},
	description={Human Connectome Project}
}

\newglossaryentry{bids}%
{
	name={BIDS},
	first={BIDS (Brain Imaging Data Structure, \cite{gorgolewski_brain_2015})},
	long={Brain Imaging Data Structure},
	description={Brain Imaging Data Structure}
}

\newglossaryentry{bedpostx}%
{
	name={BEDPOSTX},
	first={BEDPOSTX (Bayesian Estimation of Diffusion Parameters Obtained using Sampling Techniques modelling crossing --X-- fibres, \cite{jbabdi_modelbased_2012})},
	description={Bayesian Estimation of Diffusion Parameters Obtained using Sampling Techniques modelling crossing --X-- fibres}
}
\newacronym[first={FAST (FMRIB's Automated Segmentation Tool, \cite{zhang_segmentation_2001})}]%
{fast}{FAST}{FMRIB's Automated Segmentation Tool}
\newacronym[first={FIRST (FMRIB's Integrated Registration and Segmentation Tool, \cite{patenaude_bayesian_2011})}]%
{first}{FIRST}{FMRIB's Integrated Registration and Segmentation Tool}

\makeglossaries


\providecommand{\diffantom}{\emph{Diffantom}}
\providecommand{\Diffantom}{\emph{Diffantom}}
\newcommand{\lowb}{\textit{b0}}
\newcommand{\e}[1]{\ensuremath{\;\cdot\,\text{10}^\text{#1}}}
\newcommand{\vmaps}{\ensuremath{\{\mathbf{V}_i \,|\, i \in \{1,2,3\}\}}}
\newcommand{\fmaps}{\ensuremath{\{F_i \,|\, i \in \{1,2,3\}\}}}
\newcommand{\tmaps}{\ensuremath{\{T_j \,|\, j \in \{1,\ldots,5\}\}}}
\newcommand{\diffunits}{\ensuremath{\text{mm}^2\text{s}^{\text{-1}}}}


\begin{document}
\onecolumn
\firstpage{1}

\title[Supplementary Material]{\helvetica{Supplementary Material to: Diffantom}} %Please insert the title of your article here

\author[\firstAuthorLast ]{\Authors} %This field will be automatically populated
\correspondance{} %This field will be automatically populated

\extraAuth{}% If there are more than 1 corresponding author, comment this line and uncomment the next one.
%\extraAuth{Corresponding Author2: email2@uni2.edu}

\maketitle

\section*{Appendix}\label{sec:appendix}
Let $\{T'_j\}$ be the set of original fractions maps obtained with \path{act_anat_prepare_fsl}, a
  tool in \emph{MRTrix} that combines \gls*{fast} and \gls*{first}
  to generate the macrostructural 5TT map.
FA denotes the \gls*{fa} map obtained from the original \gls*{dmri} data: the local fiber orientation maps
  $\{\mathbf{V}_i\}$ with their estimated volume fractions $\{F'_i\}$ calculated with \gls*{bedpostx}.
The final $\{T_j\}$ maps of isotropic fractions are computed as follows:
  \begin{align*}
  T_1 &= (1.0-f_{cgm}) \cdot T'_1 \\
  T_2 &= (1.0-f_{dgm}) \cdot T'_2 \\
  T_3 &= (1.0-f_{wm}) \cdot T'_3 \\
  T_4 &= T'_4 \\
  T_5 &= 0.0
  \end{align*}
where $f_{\{cgm, dgm, wm\}}$ are the fractions of restricted diffusion for each tissue.
\cite{sepehrband_brain_2015} found out that the fiber fraction ranges across the corpus
  callosum from the 70$\pm$8\% in its body to an upper bound of 80$\pm$11\% in the splenium.
Therefore, we choose $f_{wm} =$ 80\% as default fraction of restricted diffusion in the
  \gls*{wm}.
To our knowledge, restricted diffusion fractions have been studied only for \gls*{wm}.
Therefore, we set $f_{cgm} =$ 25\% and $f_{dgm} =$ 50\% as they yield plausible \gls*{fa}
  and \gls*{adc} maps, assessed visually.
The final $\{F_i\}$ maps are computed as follows:
\begin{align*}
F_1 &= f_{wm} \cdot T_2 \cdot \text{FA} + w_{f1} (f_{cgm} \cdot T_1 + f_{dgm} \cdot T_2) \\
F_2 &= f_{wm} \cdot T_2 - (F_1 + F_3) + w_{f2} (f_{cgm} \cdot T_1 + f_{dgm} \cdot T_2) \\
F_3 &= f_{wm} \cdot F'_3 + w_{f3} (f_{cgm} \cdot T_1 + f_{dgm} \cdot T_2)
\end{align*}
where $w_{\{f1, f2, f3\}}$ are the contributions of the \gls*{gm} compartments to each fiber population.
By default: $w_{f1} = $ 48\%, $w_{f2} = $ 37\%, $w_{f3} = $ 15\%.
Finally, the resulting maps are normalized to fulfill $\sum_j T_j + \sum_i F_i = 1.0$.

\newpage
\bibliographystyle{frontiers/frontiersinSCNS_ENG_HUMS}
%\bibliographystyle{frontiersinHLTH&FPHY} % for Health and Physics articles
\bibliography{Remote}

\end{document}


% \documentclass[english]{frontiers/frontiers_suppmat} % for Science, Engineering and Humanities and Social Sciences articles
% \usepackage{xifthen}
% \newboolean{review}
% \setboolean{review}{true}

% \usepackage[mode=buildnew]{standalone}
% \usepackage{tikz}
% \usepackage[framemethod=TikZ]{mdframed}
% %\setcitestyle{square}
% \usepackage{url,microtype}
% \usepackage[right]{lineno}
% \usepackage[toc,nomain,acronym,shortcuts,translate=false]{glossaries}
% \usepackage[colorlinks=true,linkcolor=black, citecolor=black!80, urlcolor=black!80]{hyperref}
% \usepackage{doi}
% \usepackage[onehalfspacing]{setspace}

% \def\keyFont{\fontsize{7}{9}\helveticabold }


% \begin{document}
% \firstpage{1}
% \title[Diffantom]{Diffantom: whole-brain diffusion MRI phantoms derived from real datasets of the \acrlong{hcp}}
% \maketitle


% \end{document}
